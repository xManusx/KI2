\documentclass[fleqn,12pt]{scrartcl}
\usepackage[utf8]{inputenc}
\usepackage{color}
\usepackage[ngerman]{babel}
\usepackage{amssymb}
\usepackage{amsthm}
\usepackage{amsmath}
\usepackage{gauss}
\usepackage{braket}
\usepackage{hyperref}
\usepackage{csquotes}
\usepackage{wasysym}
\usepackage{tikz}
\usetikzlibrary{intersections}
\usetikzlibrary{arrows}
\tikzset{
	treenode/.style = {align=center, inner sep=0pt, text centered,
				font=\sffamily},
					arn_n/.style = {treenode, circle, white, font=\sffamily\bfseries, draw=black,
							fill=black, text width=1.5em},% arbre rouge noir, noeud noir
								arn_r/.style = {treenode, circle, red, draw=red, 
										text width=1.5em, very thick},% arbre rouge noir, noeud rouge
											arn_x/.style = {treenode, rectangle, draw=black,
													minimum width=0.5em, minimum height=0.5em}% arbre rouge noir, nil
													}

\usepackage{scrpage2}
\pagestyle{scrheadings}
\clearscrheadfoot
\ohead{\pagemark}
\ihead{Magnus Berendes, 21752155}
\ifoot{\today}
\ofoot{\blattn}
%\setheadtopline{1pt}
\setheadsepline{0.4pt}
\setfootsepline{0.4pt}

\usepackage{enumitem}
\newcommand{\id}{\, \mathrm{d}}
\newcommand{\intl}{\int\displaylimits}
% New definition of square root:
% it renames \sqrt as \oldsqrt
\let\oldsqrt\sqrt
% it defines the new \sqrt in terms of the old one
\def\sqrt{\mathpalette\DHLhksqrt}
\def\DHLhksqrt#1#2{%
	\setbox0=\hbox{$#1\oldsqrt{#2\,}$}\dimen0=\ht0
	\advance\dimen0-0.2\ht0
	\setbox2=\hbox{\vrule height\ht0 depth -\dimen0}%
{\box0\lower0.4pt\box2}}

\newcommand{\karos}[2]{
	\begin{tikzpicture}
		\draw[step=0.5cm, color=gray] (0,0) grid (#1 cm , #2 cm);
	\end{tikzpicture}
}
\newcommand{\abs}[1]{
	\left \vert #1 \right \vert
}
\newcommand{\absbb}[1]{
	\left \Vert #1 \right \Vert
}

\usepackage{microtype}
\usepackage[detect-weight=true, binary-units=true]{siunitx}

%TODO
\newcommand{\blattn}{Exercise 8}
\begin{document}
\section*{\blattn}
\setcounter{section}{8}
\subsection{Home Decisions}
\begin{enumerate}
	\item
		\begin{itemize}
			\item  House NR:
				\begin{align*}
					I(Nr) = \sum_{i=1}^5 -\frac15 \log_2(\frac15) = \SI{2.32}{\bit}
				\end{align*}

			\item
				Furniture:
				\begin{align*}
					I(Furn) = I(\langle 0.4, 0.6 \rangle) = -0.4\cdot \log_2(0.4) - 0.6 \cdot \log_2(0.6) = \SI{0.97}{\bit}
				\end{align*}
			\item
				Nr. of rooms:
				\begin{align*}
					I(rooms) = I(\langle 0.4, 0.6 \rangle) = \SI{0.97}{\bit}
				\end{align*}

			\item
				New kitchen:
				\begin{align*}
					I(kitch) = I(\langle 0.2, 0.8 \rangle) = -0.2 \cdot \log_2(0.2) - 0.8 \cdot \log_2(0.8) = \SI{0.72}{\bit}
				\end{align*}

			\item
				Acceptable:
				\begin{align*}
					I(acc) = I(\langle 0.4, 0.6 \rangle) = \SI{0.97}{\bit}
				\end{align*}
		\end{itemize}

	\item After the first step, all examples have the same classification (as every number only exists once), thus we simply return the corresponding classification



		\begin{tikzpicture}
			[
				grow                    = down,
								sibling distance        = 6em,
										level distance          = 10em,
												edge from parent/.style = {draw, -latex},
														every node/.style       = {font=\footnotesize, draw, shape=circle}
																	]
			\node {NR}
			child { node [shape=rectangle]{TRUE} edge from parent node [left,draw=none] {$1$}
			}
			child { node [shape=rectangle]{FALSE} edge from parent node [left,draw=none] {$2$}}
			child { node [shape=rectangle]{TRUE} edge from parent node [left,draw=none] {$3$}}
			child { node [shape=rectangle]{FALSE} edge from parent node [left,draw=none] {$4$}}
			child { node [shape=rectangle]{TRUE} edge from parent node [left,draw=none] {$5$}};
			%child { node {aligned at}
				%child { node {relation sign} }
				%child { node {several places} }
				%child { node {center} } }
			%child { node {first left,\\centered,\\last right} } };
		\end{tikzpicture}

	\item
		\begin{align*}
			Gain(nr) &= I\left(\left\langle\frac35, \frac25\right\rangle\right) - \sum_i \frac{p_i + n_i}{p+n} I\left(\left\langle \frac{p_i}{p_i + n_i}, \frac{n_i}{p_i + n_i}\right\rangle\right)\\
			&= \SI{0.97}{\bit} - \sum_i \frac{p_i + n_i}{p+n} \cdot I(\langle 1, 0\rangle)\qquad (\text{or } I(\langle 0,1\rangle))\\
			&= \SI{0.97}{\bit} - 0 = \SI{0.97}{\bit}
		\end{align*}
		
\end{enumerate}

\subsection{Home Decisions 2.0}
Similar to the decision tree in 8.1.2, the tree splits up by \enquote{Name}, into 8 leafs, containing the corresponding result.
Imagining the tree is left as an excercise to the reader.


\end{document}
