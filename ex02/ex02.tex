\documentclass[fleqn,12pt]{scrartcl}
\usepackage[utf8]{inputenc}
\usepackage{color}
\usepackage[ngerman]{babel}
\usepackage{csquotes}
\usepackage{amssymb}
\usepackage{amsthm}
\usepackage{amsmath}
\usepackage{gauss}
\usepackage{braket}
\usepackage{hyperref}
\usepackage{wasysym}
\usepackage{pgf}
\usepackage{tikz}
\usetikzlibrary{positioning,shapes,arrows}
%\usetikzlibrary{intersections}
%\usetikzlibrary{arrows}
%\tikzset{
	%treenode/.style = {align=center, inner sep=0pt, text centered,
				%font=\sffamily},
					%arn_n/.style = {treenode, circle, white, font=\sffamily\bfseries, draw=black,
							%fill=black, text width=1.5em},% arbre rouge noir, noeud noir
								%arn_r/.style = {treenode, circle, red, draw=red, 
										%text width=1.5em, very thick},% arbre rouge noir, noeud rouge
											%arn_x/.style = {treenode, rectangle, draw=black,
													%minimum width=0.5em, minimum height=0.5em}% arbre rouge noir, nil
													%}

\usepackage{graphicx}
\usepackage{scrpage2}
\pagestyle{scrheadings}
\clearscrheadfoot
\ohead{\pagemark}
\ihead{Magnus Berendes, 21752155}
\ifoot{\today}
\ofoot{\blattn}
%\setheadtopline{1pt}
\setheadsepline{0.4pt}
\setfootsepline{0.4pt}

\usepackage{enumitem}
\newcommand{\id}{\, \mathrm{d}}
\newcommand{\intl}{\int\displaylimits}
% New definition of square root:
% it renames \sqrt as \oldsqrt
\let\oldsqrt\sqrt
% it defines the new \sqrt in terms of the old one
\def\sqrt{\mathpalette\DHLhksqrt}
\def\DHLhksqrt#1#2{%
	\setbox0=\hbox{$#1\oldsqrt{#2\,}$}\dimen0=\ht0
	\advance\dimen0-0.2\ht0
	\setbox2=\hbox{\vrule height\ht0 depth -\dimen0}%
{\box0\lower0.4pt\box2}}

\newcommand{\karos}[2]{
	\begin{tikzpicture}
		\draw[step=0.5cm, color=gray] (0,0) grid (#1 cm , #2 cm);
	\end{tikzpicture}
}
\newcommand{\abs}[1]{
	\left \vert #1 \right \vert
}
\newcommand{\absbb}[1]{
	\left \Vert #1 \right \Vert
}

\usepackage{microtype}
%TODO
\newcommand{\blattn}{Exercise 2}
\begin{document}
\section*{\blattn}
\setcounter{section}{2}

\subsection{Nuclear Test}
		\begin{align*}
			X = \Set{N, E, S_1, S_2}\\
		\end{align*}
\begin{figure}[h]
	\centering
  \includegraphics[width=\textwidth]{a.png}
	\caption{Network (a)}
	\label{fig1}
\end{figure}

\begin{itemize}
	\item
		No parents for $N$ obviously
	\item
		If we assume that a nuclear test does not cause a real, \enquote{organic} earthquake, then there's no relationship between the two, thus $E$ also has no parents
	\item
		$S_1$ and $S_2$ obviously depend on $N$ and $E$, both can cause the seismometers to go off
\end{itemize}
\begin{figure}[h]
	\centering
  \includegraphics[width=\textwidth]{b.png}
	\caption{Network (b)}
	\label{fig2}
\end{figure}
\begin{itemize}
	\item
		No parents for $S_1$ obviously
	\item
		Also no parents for $S_2$ as we don't want both symptoms to be dependent of each other
	\item
		$E$ and $N$ are probable causes for $S_1$ and $S_2$
\end{itemize}

\subsection{Medical Bayesian Network}


 \begin{tikzpicture}[
   node distance=1cm and 1cm,
   mynode/.style={draw,circle,text width=1cm,align=center}
 ]
 \node[mynode] (smoke) {Smoke};
 \node[mynode,right=of smoke] (asia) {Asia};
 \node[mynode,below=of smoke] (tbc) {TBC};
 \node[mynode,left =of tbc] (lc) {LC};
 \node[mynode,below =of asia] (bron) {bron};
 \node[mynode,below =of tbc] (xray) {Xray};
 \node[mynode,below =of lc] (dysp) {Dysp};

 \path (smoke) edge[-latex] (lc)
 (asia) edge[-latex] (tbc)
 (smoke) edge[-latex] (bron)
 (lc) edge[-latex] (xray)
 (tbc) edge[-latex] (dysp)
 (bron) edge[-latex] (dysp)
 (lc) edge[-latex] (dysp)
 (tbc) edge[-latex] (xray);
 \end{tikzpicture}


 \subsection{Medical Bayesian Network 2}
 \begin{enumerate}
	 \item$\quad$

		 \begin{tikzpicture}[
			 node distance=1cm and 1cm,
			 mynode/.style={draw,circle,text width=1cm,align=center}
			 ]
			 \node[mynode] (hbt) {HBT};
			 \node[mynode, above left=of hbt] (mal) {Mal};
			 \node[mynode, above right=of hbt] (men) {Men};
			 \node[mynode, below=of hbt] (f){F};


			 \path (mal) edge[-latex] (hbt)
				 (men) edge[-latex] (hbt)
				 (hbt) edge[-latex] (f);
		 \end{tikzpicture}
 And associated CPTs
 \item
	 \begin{align*}
		 P(mal, \neg men | f) &= \frac{P(mal, \neg men, f)}{P(f)}\\
													&= \frac{\sum_{HBT, F} P(mal, \neg men, f, hbt)}{\sum_{Mal, Men, HBT} P(hbt, f, mal, men)}\\
							 &= \frac{\sum_{HBT, F} P(mal) \cdot P(\neg men) \cdot P(hbt | mal, \neg men) \cdot P(f | hbt)}{\sum_{Mal, Men, HBT} P(mal) \cdot P(men) \cdot P(f | hbt) \cdot P(hbt | mal, men)}\\
							 &= \frac{0.04 \cdot\sum_{HBT, F}  P(hbt | mal, \neg men) \cdot P(f | hbt)}{\sum_{Mal, Men, HBT} P(mal) \cdot P(men) \cdot P(f | hbt) \cdot P(hbt | mal, men)}\\
					&= \frac{0.04 \cdot (0.1 \cdot 0.998 + 0.002 \cdot 0.1 + 0.9 \cdot 0.05 + 0.9 \cdot 0.95)}{666}\\
				 &= \frac{0.04}{0.170429} \approx 23.5\%
	 \end{align*}
 \end{enumerate}
%$
%hbt f
 %0 0 0.1 \cdot 0.998
 %0 1 0.002 \cdot 0.1
 %1 0 0.9 \cdot 0.05
 %1 1 0.9 \cdot 0.95
 %$



 \end{document}
