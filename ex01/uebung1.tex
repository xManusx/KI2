\documentclass[fleqn,12pt]{scrartcl}
\usepackage[utf8]{inputenc}
\usepackage{color}
%\usepackage[]{babel}
\usepackage{csquotes}
\usepackage{amssymb}
\usepackage{amsthm}
\usepackage{amsmath}
\usepackage{gauss}
\usepackage{braket}
\usepackage{hyperref}
\usepackage{wasysym}
\usepackage{tikz}
\usetikzlibrary{intersections}
\usetikzlibrary{arrows}
\tikzset{
	treenode/.style = {align=center, inner sep=0pt, text centered,
				font=\sffamily},
					arn_n/.style = {treenode, circle, white, font=\sffamily\bfseries, draw=black,
							fill=black, text width=1.5em},% arbre rouge noir, noeud noir
								arn_r/.style = {treenode, circle, red, draw=red, 
										text width=1.5em, very thick},% arbre rouge noir, noeud rouge
											arn_x/.style = {treenode, rectangle, draw=black,
													minimum width=0.5em, minimum height=0.5em}% arbre rouge noir, nil
													}

\usepackage{scrpage2}
\pagestyle{scrheadings}
\clearscrheadfoot
\ohead{\pagemark}
\ihead{Magnus Berendes, 21752155}
\ifoot{\today}
\ofoot{\blattn}
%\setheadtopline{1pt}
\setheadsepline{0.4pt}
\setfootsepline{0.4pt}

\usepackage{enumitem}
\newcommand{\id}{\, \mathrm{d}}
\newcommand{\intl}{\int\displaylimits}
% New definition of square root:
% it renames \sqrt as \oldsqrt
\let\oldsqrt\sqrt
% it defines the new \sqrt in terms of the old one
\def\sqrt{\mathpalette\DHLhksqrt}
\def\DHLhksqrt#1#2{%
	\setbox0=\hbox{$#1\oldsqrt{#2\,}$}\dimen0=\ht0
	\advance\dimen0-0.2\ht0
	\setbox2=\hbox{\vrule height\ht0 depth -\dimen0}%
{\box0\lower0.4pt\box2}}

\newcommand{\karos}[2]{
	\begin{tikzpicture}
		\draw[step=0.5cm, color=gray] (0,0) grid (#1 cm , #2 cm);
	\end{tikzpicture}
}
\newcommand{\abs}[1]{
	\left \vert #1 \right \vert
}
\newcommand{\absbb}[1]{
	\left \Vert #1 \right \Vert
}

\usepackage{microtype}
%TODO
\newcommand{\blattn}{Exercise 1}
\begin{document}
\section*{\blattn}
\setcounter{section}{1}
\begin{itemize}
	\item
2.3 zuerst machen könnte hilfreich sein
\item
$$
p(a|b) = \frac{p(b|a)\cdot p(a)}{p(b)}
$$

\item
$$p(H|e) = \frac{p(e|H \cdot p(H)}{p(e)}
$$
wie wahrscheinlich ist es, dass es stimmt, was ich glaube über die Welt zu wissen (Hypothese), gegeben meine Wahrnehmung (evidenz)

\item
$p(H)$ prior, $p(H|e)$ posterior $\Rightarrow$ $p(H)$ im nächsten Schritt

\item
prior muss geschätzt werden, $p(e|H)$ sollte aus H klar sein. $p(e)$ ist schwer zu bestimmen
\item
	$$p(e) = p(e|H) \cdot p(H) + p(e|\neg H) \cdot p(\neg H)$$
	$$p(e|H) \cdot p(H) = p(e\wedge H)$$
\item
	$$P(down | F=age25, positiv) = P(down | positiv)$$ weil conditional independence
\item
	$$P(a\wedge b) = 1 - p(\neg a \vee \neg B) = 1 - (1 -  \dots ) )$$
\end{itemize}

\subsection{AFT Tests}

\begin{enumerate}
	\item
\begin{align*}
	P(down | F=Age_{25}) &= \frac1{1250}\\
	P(down | F=Age_{43}) &= \frac1{50}\\
	P(pos | \neg down) &= 0.01\\
	P(\neg pos | down) &= 0.01\\
	P(pos | down) &= 0.99\\
	P(pos, down) &= P(pos | down) \cdot P(down)\\
\end{align*}
\item
	\begin{align*}
		P(down | F=Age_{25}, pos) &= P(down | pos) \\
																											 &= \frac{P(pos|down)\cdot P(down)}{P(pos)}\\
															%&= \frac{P(pos,down) \cdot P(down)}{P(pos) \cdot P(down)}\\
				 &= \frac{P(pos|down)\cdot P(down)}{P(pos|down)\cdot P(down) + P(pos| \neg down) \cdot P(\neg down)}\\
																&= \frac{P(pos|down)\cdot P(down)}{P(pos|down)\cdot P(down) + P(pos| \neg down) P(\neg down)}\\
	\end{align*}

\end{enumerate}

\subsection{Stochastic Wumpus}
\begin{align*}
	&P(W\not =F_1) = 0.9\\
	&P(W = F_1 = 0.1\\
	&P(W = F_2) = 0.6\\
	&P(S_x | W = F_x) = 1\\
	&P(S_x | W \not = F_x) = 0.2
\end{align*}
\begin{enumerate}
	\item
		\begin{align*}
			P(W=F_1 | S_1) &= \frac{P(S_1 | W=F_1) \cdot P(W=F_1)}{P(S_1)}\\
									 &= \frac{1\cdot 0.1}{P(S_1)}\\
			P(S_1) &= P(S_1 | W=F_1) \cdot P(W=F_1) + P(S_1 | W \not = F_1) \cdot P(W \not = F_1)\\
						 &= 1\cdot 0.1 + 0.2 \cdot 0.9 = 0.28\\
			P(W= F_1| S_1) &= \frac{1\cdot 0.1}{0.28} = 0.36\\
														&\Rightarrow P(W \not = f_1) = 0.64
		\end{align*}
		The probability for $W=F_2$ can't be more than 0.64 as the Wumpus can not be in more than one place. As the probability for $W=F_1$ increases, we can assume that the probability for $W=F_2$ decreases as $\sum_{w\in W} P(w) = 1$
	\item
		$P(W= F_2)$ increases, $P(W= F_1)$ decreases again
\end{enumerate}
\subsection{Disjunctive Random Variables}
\begin{align*}
	P(A \vee B \vee C) &= P(A \vee (B\vee C))\\
										 &=P(A) + P(B\vee C) - P(A \wedge (B\vee C))\\
	&= P(A) + P(B) + P(C) - P(B\wedge C) - P(A \wedge (B\vee C))\\
																	 &= P(A) + P(B) + P(C) - P(B\wedge C) - P((A \wedge B) \vee (A\wedge C))\\
																	 &= P(A) + P(B) + P(C) - P(B\wedge C) - (P(A\wedge B) + P(A\wedge C) \\ &\quad - P((A\wedge B) \wedge (A\wedge B)))\\
																	 &= P(A) + P(B) + P(C) - P(B\wedge C) - P(A\wedge B) - P(A\wedge C) + P(A\wedge B \wedge C)\\
\end{align*}
\subsection{Chained Production Elements}
\begin{enumerate}
	\item
\begin{align*}
	P(works) &= P(a\wedge b \vee c\wedge d \vee e\wedge f)\\
							 &= P(a, b) + P(c, d) + P (e, f) \\ &\quad- P(a,b,c,d) - P(c,d,e,f) - P(a,b,e,f) + P(a,b,c,d,e,f)\\
								 &=0.95\cdot 0.90 + 0.85 \cdot 0.80 + 0.75\cdot0.70 \\
					 &\quad- 0.95\cdot0.90\cdot0.85\cdot0.80 - 0.85\cdot0.80\cdot0.75\cdot0.70\\
								 &\quad- 0.95\cdot0.90\cdot0.75\cdot0.70 + 0.95\cdot0.90\cdot0.85\cdot0.80\cdot0.75\cdot0.70 \\
					 &=0.978
\end{align*}
\item
	%\begin{align*}
		%&a \quad&\quad b \quad&\quad c \quad&\quad d \quad&\quad e \quad&\quad f&\\
		%\quad&\quad.95 \quad&\quad .90 \quad&\quad .85 \quad&\quad .80 \quad&\quad .75 \quad&\quad .70\quad&\quad
	%\end{align*}
	\begin{align*}
		a\wedge b \wedge c \wedge e &:= x\\
		P(x) &= 0.95*0.90*0.85*0.75 = 0.545\\
		c \wedge d \wedge a \wedge f &:= y\\
		P(y) &= 0.85 * 0.8 * 0.95 * 0.70 = 0.452\\
		e \wedge f \wedge d \wedge b &= z\\
		P(z) &= 0.75 * 0.70 * 0.80 * 0.90 = 0.378\\
	\end{align*}
	\begin{align*}
		P(works) &= P(a\wedge b\wedge c\wedge e \vee c\wedge d \wedge a \wedge f \vee e\wedge f\wedge d \wedge b)\\
										&= P(x \vee y \vee z)\\
						 &= P(x) + P(y) + P(z) - P(y \wedge z) - P(x \wedge y) - P(x \wedge z) + P(x \wedge y \wedge z)\\
						 &= 1.375 - 0.171 - 0.246 - 0.206 + 0.093\\
						 &= 0.845
	\end{align*}

	This works because the events are still independent of each other. Each \enquote{production lane} now just needs more elements to work.
\end{enumerate}
\end{document}
