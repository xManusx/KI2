\documentclass[fleqn,12pt]{scrartcl}
\usepackage[utf8]{inputenc}
\usepackage{color}
\usepackage[ngerman]{babel}
\usepackage{amssymb}
\usepackage{amsthm}
\usepackage{amsmath}
\usepackage{gauss}
\usepackage{braket}
\usepackage{hyperref}
\usepackage{wasysym}
\usepackage{pgf}
\usepackage{tikz}
\usetikzlibrary{positioning,shapes,arrows}
%\usetikzlibrary{intersections}
%\usetikzlibrary{arrows}
%\tikzset{
	%treenode/.style = {align=center, inner sep=0pt, text centered,
				%font=\sffamily},
					%arn_n/.style = {treenode, circle, white, font=\sffamily\bfseries, draw=black,
							%fill=black, text width=1.5em},% arbre rouge noir, noeud noir
								%arn_r/.style = {treenode, circle, red, draw=red, 
										%text width=1.5em, very thick},% arbre rouge noir, noeud rouge
											%arn_x/.style = {treenode, rectangle, draw=black,
													%minimum width=0.5em, minimum height=0.5em}% arbre rouge noir, nil
													%}

\usepackage{scrpage2}
\pagestyle{scrheadings}
\clearscrheadfoot
\ohead{\pagemark}
\ihead{Magnus Berendes, 21752155}
\ifoot{\today}
\ofoot{\blattn}
%\setheadtopline{1pt}
\setheadsepline{0.4pt}
\setfootsepline{0.4pt}

\usepackage{enumitem}
\newcommand{\id}{\, \mathrm{d}}
\newcommand{\intl}{\int\displaylimits}
% New definition of square root:
% it renames \sqrt as \oldsqrt
\let\oldsqrt\sqrt
% it defines the new \sqrt in terms of the old one
\def\sqrt{\mathpalette\DHLhksqrt}
\def\DHLhksqrt#1#2{%
	\setbox0=\hbox{$#1\oldsqrt{#2\,}$}\dimen0=\ht0
	\advance\dimen0-0.2\ht0
	\setbox2=\hbox{\vrule height\ht0 depth -\dimen0}%
{\box0\lower0.4pt\box2}}

\newcommand{\karos}[2]{
	\begin{tikzpicture}
		\draw[step=0.5cm, color=gray] (0,0) grid (#1 cm , #2 cm);
	\end{tikzpicture}
}
\newcommand{\abs}[1]{
	\left \vert #1 \right \vert
}
\newcommand{\absbb}[1]{
	\left \Vert #1 \right \Vert
}

\usepackage{microtype}
%TODO
\newcommand{\blattn}{Exercise 5}
\begin{document}
\section*{\blattn}
\setcounter{section}{5}

\subsection{Decision Networks}
\begin{enumerate}
	\item$\quad$\\

 \begin{tikzpicture}[
   %node distance=1cm and 1cm,
   mynode/.style={draw,circle,text width=5em,align=center},
   mysquare/.style={draw,rectangle,text width=5em,minimum height=5em,text centered},
   mydi/.style={draw,diamond,text width=5em,align=center},
	 node distance=2cm and 3cm,
 ]
	 \node[mysquare] (b){B};
	 \node[mynode, right=of b](m){M};
	 %above right=0.7cm and 4cm of A]
	 %\node[mysquare, above right=2cm and 0.8cm of b](buy){Buy};
	 \node[mynode, below=of b](p){P};
	 \node[mydi, below left = 0.6cm and 3cm of b](u){U};

	 \path %(buy) edge[-latex, bend right] (b)
	 %(buy) edge[-latex, bend left] (m)
	 (b) edge[-latex, bend right] (u)
	 (b) edge[-latex] (p)
	 (b) edge[-latex] (m)
	 (m) edge[-latex, bend left] (p)
	 (p) edge[-latex,bend left] (u);
 \end{tikzpicture}

 With associated CPTs as given in the assignment.
		\item
			\begin{align*}
				B &= \top\\
				P(m) &= 0.9\\
				P(\neg m) &= 0.1\\
				P(p) &= P(p|b,m) \cdot P(b,m) + P(p|b,\neg m)\cdot P(b,\neg m) \\
				%&\quad+ P(p|\neg b,M) \cdot P(\neg b, M) + P(p|\neg b,\neg m) \cdot P(\neg b, \neg m)\\
				&= 0.9 \cdot 0.9 + 0.5 \cdot 0.1 + 0 + 0\\
				&= 0.86\\
				P(\neg p) &= 0.14\\
				EU(b) &= 0.86 \cdot 2000 - 100 = 1620 %+ (0.14 * 2000) - 100 \\
			\end{align*}

			\begin{align*}
				B &= \bot\\
				P(m) &= 0.7\\
				P(\neg m) &= 0.3\\
				P(p) &= P(p|\neg b,m) \cdot P(\neg b,m) + P(p|\neg b,\neg m)\cdot P(\neg b,\neg m) \\
				&= 0.8 \cdot 0.7 + 0.3\cdot 0.3 = 0.65\\
				P(\neg p) &= 0.35\\
				EU(\neg b) &= 0.65 \cdot 2000 - 100 = 1200
			\end{align*}
\end{enumerate}

\subsection{Allais Paradox}
\begin{align*}
	A &= [0.8, 4000; 0.2, 0]\\
	EU(A) &= 3200\\
	B &= [1, 3000; 0, 0]\\
	EU(B) &= 3000\\
	C &= [0.2, 4000; 0.8, 0]\\
	EU(C) &= 800\\
	D &= [0.25, 3000; 0.75, 0]\\
	EU(D) &= 750\\
\end{align*}
\begin{enumerate}
	%\item
%From $B \prec A$ we conclude: $U(3000) > 0.8U(4000) + 0.2U(0)$
%\item
	%From $C \prec D$ we conclude $0.2U(4000) + 0.8U(0)> 0.25U(3000) + 0.75U(0)$
\item From the definition:

	$B \prec A \Rightarrow [p, B;(1-p),D] \prec [p, A; (1-p), D]$
\item
	As this should be valid for any $p$, set $p$ to 0.5:
	\begin{align*}
		U([p,B;(1-p),D]) &= p\cdot 3000 + (1-p)750 = 1500 + 375 = 1875\\
		U([p,A;(1-p),D]) &= p\cdot 3200 + (1-p)750 = 1600 + 375 = 1975\\
	\end{align*}

\item
	The following should be valid from the definition of $\prec$

	$[p,B;(1-p),D]) \prec [p,A;(1-p),D] \Leftrightarrow U([p,B;(1-p),D]) > U([p,A;(1-p),D])$

\item
	But, as shown in 2: $U([p,B;(1-p),D]) < U([p,A;(1-p),D])$ which contradicts 3.

	%\begin{align*}
		%A \prec B &\Rightarrow [p, A; (1-p) C] \prec [p, B; (1-p) C]\\
		%&\Rightarrow U(p\cdot E(A)) + U((1-p)\cdot E(C)) > U(p\cdot E(B)) + U((1-p) \cdot E(C))\\
		%&\Rightarrow U(p\cdot3200) + U((1-P) \cdot 800) > U(p\cdot 3000) + U((1-p) \cdot 800)\\
		%&\Rightarrow U(p\cdot3200) > U(p\cdot 3000)\\
	%\end{align*}
\end{enumerate}

\end{document}
